\documentclass[a4paper]{article}

% Encoodaus, joka sopii suomenkielellä (esim. ä ja ö)
\usepackage[utf8]{inputenc}
\usepackage[T1]{fontenc}

% Suomenkielinen tavutus
\usepackage[finnish]{babel}

% Viitteet
\usepackage{natbib}

% Otsikkojen päätteetön fontti
\usepackage{sectsty}
\allsectionsfont{\sffamily\large}

% Viitteiden merkit
\bibpunct{(}{)}{;}{a}{,}{,}

% page counting, header/footer
\usepackage{fancyhdr}
\usepackage{lastpage}
\usepackage{ifthen}
\pagestyle{fancy}
\lhead{\footnotesize \parbox{11cm}{} }
\rhead{\footnotesize \parbox{11cm}{} }
\lfoot{\footnotesize \parbox{11cm}{\ifthenelse{\thepage=\pageref{LastPage}}{Created with Dippa Editor \\ http://mkos.futupeople.com/dippa}{}}}
\renewcommand{\headrulewidth}{0.0pt}

\begin{document}

\title{\small T-76.5612 Software Project Management \\ Vierailuluento \\ \huge NSN \& Reaktor}
\date{27.2.2012}
\author{Mikko Koski \\ mikko.koski@aalto.fi \\ 66467F}
\maketitle

\normalsize

\section{Johdanto}

Luentovierailu suuntaitui tällä kertaa ohjelmistoyritys Reaktorin tiloihin. Luennoitsijoina toimivat Nokia Siemens Networkistä Maija-Riitta Varis sekä Reaktorilta Laura Snellman-Junna. Variksen luennon aiheena oli johtajuus, johtaminen sekä tiimihengen luonti ja Snellman-Junnann aiheena Agile-valmennus, itseohjautuvat tiimit ja johtajuus ketterässä kehityksessä.

\section{Maija-Riitta Varis}

Maija-Riitta Varis kertoi luennolla kokemuksiaan toimimisesta ihmisten johtajana Nokia Siemens Networksillä. Luennon alkupuolella Varis puhui muutoksista, joita johtaja joutuu kohtaamaan työelämässä. Hän korosti, että muutokselle pitää aina olla järkevä syy, jotta muutoksen pystyy perustelemaan alaisilleen ja jotta alaiset voi motivoida muutoksen taakse.

Varis korosti luennollaan myös tunneälyn merkitystä. Hän suositteli aiheesta paria kirjaa sekä kertoi itse käyttävänsä tunneälyä johtamistehtävissä, esimerkiksi tiimijäsenten valinnassa. Tämä oli mielestäni mielenkiintoista kuultavaa, sillä minusta tuntuu, että Varis suhtautuu ihmisten johtamiseen hieman eri tavalla kuin mitä itse suhtautuisin. Pidän itseäni ääriäni myöten analyyttisenä ihmisenä, joka tekee päätökset rationaalisten faktojen perusteella. Tiiminjäsenten valinnassa luottaisin itse faktoihin ja tiiminjäsenten kompetensseihin. Tällainen faktoihin pohjautuva tiiminjäsenten valinta saattaa tuottaa tuloksena tiimin, joka on kyvyiltään tähtitiimi, mutta jonka henkilökemiat eivät sopi yhteen keskeneen. Olen varma, että itselläni on paljon kehitettävää nimenomaan tunneälyn puolella.

Varis nosti luennollaan esille myös palkitsemisen merkityksen. Palkitsemisella Varis ei tarkoittanut niinkään rahallisen palkinnon antamista vaan pikemminkin tunnustuksen antamista. Hän kertoi kaksi esimerkkiä yksinkertaisesta palkitsemisesta. Ensimmäisessä esimerkissä Varis lupasi ostaa ohjelmistokehittäjälle palkinnoksi kossupullon, jos tämä tekisi valmiiksi halutun ominaisuuden. Toisessa esimerkissä hän kertoi tavasta, jossa tiimi valitsi joukostaan viikon työntekijän.

Mielestäni kumpikaan esimerkeistä ei kuitenkaan ole paras mahdollinen esimerkki onnistuneesta palkitsemisesta. Erityisesti ensimmäinen esimerkki on mielestäni jopa hieman huolestuttava. Miten pääsi syntymään tilanne, jossa kehittäjän piti käytännössä "kiristää" kossupullo esimieheltään, jotta tämä suostui tekemään ominaisuuden? Oliko Varis kentien erityisen huono antamaan alaisille tunnustusta ja tästä syystä alainen joutui hieman radikaalein keinoin muistuttamaan esimiestään tästä? Vai oliko kehittäjä kenties kyvytön ymmärtämään kokonaisuutta ja näkemään ominaisuuden tärkeyttä tuotteen kannalta? Tällainen etukäteen palkitseminen ei varmastikaan voi muodostua pitkäaikaiseksi tavaksi, mutta onneksi tämä oli ymmärtääkseni yksittäistapaus.

Toisessa esimerkissä positiivista oli se, että palkinnon saajan valitsi tiimi keskuudestaan. Toisaalta se, että palkinto oli viikottainen rutiini vähentää mielestäni palkinnon merkitystä. Olen sitä mieltä, että palkitseminen toimii parhaiten silloin kun siihen on aihetta, ei silloin kun on sen aika. Itse en tästä syystä käyttäisi viikottaista palkitsemista vaan palkitsisin kun kehittäjät yltävät erityisen upeisiin saavutuksiin.

Mielestäni eräs luennon ajatuksia herättävimmistä pointeista oli Variksen toteamus, jonka mukaan ihmisiä ei voi motivoida. Alaisia ei voi siis käskeä motivoitumaan. Jokaisen on itsensä motivoitava itsensä. Se mitä organisaatio ja esimies voivat asian hyväksi tehdä, on luoda ympäristö, jossa on helppo motivoitua.

% Feelings... how they effect

% Meeting a change, need of the change
% How it's affecting me? 20 mes?

% Tunneäly

% Palkitseminen, julkinen palkitseminen

% best worker of the week?

% Can not motivate people, create an environment

\section{Laura Snellman-Junna: Agile-valmennus ja johtajuus ketterässä kehityksessä}

Snellman-Junnan osuus koostui pääosin kolmesta osasta: Agile-manifestista, itseohjautuvista tiimeistä sekä johtajuudesta ketterässä ympäristössä.

Itseohjautuva tiimi on mielenkiintoinen toimintatapa, joka sopii hyvin yhteen ketterien ohjelmistokehitysmenetelmien 
kanssa. Esimerkiksi Scrum-prosessissa, jossa tiimi antaa tietyn lupauksen työstä, jonka tiimi sitoutuu tekemään yhden iteraation aikana, on hyvä, että tiimi itseohjautuvasti päättää mihin sitoutuu. Snellman-Junnan mukaan tiimiä voi sanoa itseohjautuvaksi, jos tiimistä tulee vielä paremmin toimiva vaikka tiimi jätettäisiin toimimaan yksin.

Snellman-Junnan mukaan tutkimukset ovat osoittaneet, että itseohjautuvat tiimit tuottavat parempia tuloksia kuin perinteisen mallin mukaan ylhäältä johdetut tiimit. Tämä on varmasti totta, sillä itseohjautuvuus luo mahdollisuuksia vaikutta omaan työhön. Vaikuttamismahdollisuus omaan työhön on taas omiaan nostamaan työmotivaatiota ja korkea motivaatio taas luo hyvää tulosta.

Ketterät menetelmät painottavat yksilöiden sekä hyvien itseohjautuvien tiimien merkitystä, joten mihin sitten ketterissä menetelmissä tarvitaan johtajia? Snellman-Junnan mukaan itseohjautuvatkin tiimit tarvitsevat tukea ja johtajuutta, jotta niistä voi tulla itseohjautuvia ja jotta ne pysyvät itseohjautuvina. Tämä on mielestäni oleellinen pointti. Mielestäni parhaitenkin toimivat itseohjautuvat tiimit tarvitsevat aina silloin tällöin ulkopuolista näkemystä ja sparrausta. Tiimi saattaa jäädä lepäämään laakereillaan, eikä esimerkiksi havaitse toiminnassaan kehitettävää vaikka kehittämiskohteita olisikin. Toisaalta tiimi saattaa kohdata ongelman, jota se ei pysty itse syystä tai toisesta ratkaisemaan. Tällaisissa tilanteissa tiimin ulkopuolinen Agile-valmentaja saattaa tuoda ratkaisun tiimin ongelmaan tuomalla mukaan omaa kokemustaan ja tiimin ulkopuolista näkemystä.

Luennon lopuksi Snellman-Junna esitteli valmentajan tärkeimmät ominaisuudet ja keinot, joilla perinteisestä johtajasta voi tulla valmentaja. Mielestäni yksi tärkeimmistä Snellman-Junnan esittämistä valmentajan ominaisuuksista on se, että valmentaja irrottautuu työn tuloksesta. Vain tällä keinoin valmentaja pystyy olemaan aidosti tiimin ulkopuolinen tukija.

Nostin edellä mainitun valmentajan ominaisuuden esille siksi, koska luulen, että tätä ominaisuutta on monissa organisaatiossa vaikea saavuttaa. Organisaatiolla on nykyään tapana vyöryttää tulosvastuuta niin tiiminjohtajien kuin projektijohtajienkin vastuulle, jolloin tulosvastuutonta valmentajaa saatetaan pitää organisaation vapaamatkustajana. Ketteriä menetelmiä vähemmän käyttävälle organisaatiolle voi olla vaikea perustella, miksi tulosvastuuton valmentaja tuottaisi etuja organisaatiolle.

\section{Mitä opin?}

Molemmat luennot olivat miellyttäviä seurata. Mitään varsinaista mullistavan uutta asiaa ei kummassakaan luennossa tullut esille, mutta ne herättivät kuitenkin mielenkiintoisia ajatuksia. Erityisesti Variksen kertomuksen ja henkilökohtaiset esimerkit todellisista johtamistilanteista olivat miellyttävää kuunneltavaa. Myös Variksen ajatukset ihmisten johtamisesta ja tunneälyllä johtamisesta herättivät paljon ajatuksia. 

Snellman-Junnan luennolla käsiteltiin paljon vanhoja tuttuja käsitteitä ketteryydestä. Olisin toivonut, että luento olisi pureutunut syvällisemmin ketterään kehitykseen, tiimeihin ja valmennukseen. Mielestäni luento jäi nyt hyvin pintapuoliseksi. Olisin myös kuullut mielelläni enemmän käytännön esimerkkejä työstä, jota Reaktorin konsultit tekevät Agile-valmennuksen parissa.

% Self-organazing team

% 

% \bibliographystyle{plainnat}
% \bibliography{ref}

\end{document}