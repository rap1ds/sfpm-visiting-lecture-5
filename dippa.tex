\documentclass[a4paper]{article}

% Encoodaus, joka sopii suomenkielellä (esim. ä ja ö)
\usepackage[utf8]{inputenc}
\usepackage[T1]{fontenc}

% Suomenkielinen tavutus
\usepackage[finnish]{babel}

% Viitteet
\usepackage{natbib}

% Otsikkojen päätteetön fontti
\usepackage{sectsty}
\allsectionsfont{\sffamily\large}

% Viitteiden merkit
\bibpunct{(}{)}{;}{a}{,}{,}

% page counting, header/footer
\usepackage{fancyhdr}
\usepackage{lastpage}
\usepackage{ifthen}
\pagestyle{fancy}
\lhead{\footnotesize \parbox{11cm}{} }
\rhead{\footnotesize \parbox{11cm}{} }
\lfoot{\footnotesize \parbox{11cm}{\ifthenelse{\thepage=\pageref{LastPage}}{Created with Dippa Editor \\ http://mkos.futupeople.com/dippa}{}}}
\renewcommand{\headrulewidth}{0.0pt}

\begin{document}

\title{\small T-76.5612 Software Project Management \\ Vierailuluento \\ \huge NSN \& Reaktor}
\date{27.2.2012}
\author{Mikko Koski \\ mikko.koski@aalto.fi \\ 66467F}
\maketitle

\normalsize

\section{Johdanto}

Luentovierailu suuntaitui tällä kertaa ohjelmistoyritys Reaktorin tiloihin. Luennoitsijoina toimivat Nokia Siemens Networkistä Maija-Riitta Varis sekä Reaktorilta Laura Snellman-Junna. Variksen luennon aiheena oli johtajuus, johtaminen sekä tiimihengen luonti ja Snellman-Junnann aiheena Agile-valmennus ja johtajuus ketterässä kehityksessä.

\section{Maija-Riitta Varis}

Maija-Riitta Varis kertoi luennolla kokemuksiaan toimimisesta ihmisten johtajana Nokia Siemens Networksillä. Luennon alkupuolella Varis puhui muutoksista, joita johtaja joutuu kohtaamaan työelämässä. Hän korosti, että muutokselle pitää aina olla järkevä syy, jotta muutoksen pystyy perustelemaan alaisilleen ja jotta alaiset voi motivoida muutoksen taakse.

Varis korosti luennollaan myös tunneälyn merkitystä. Hän suositteli aiheesta paria kirjaa sekä kertoi itse käyttävänsä tunneälyä johtamistehtävissä, esimerkiksi tiimijäsenten valinnassa. Tämä oli mielestäni mielenkiintoista kuultavaa, sillä minusta tuntuu, että Varis suhtautuu ihmisten johtamiseen hieman eri tavalla kuin mitä itse suhtautuisin. Pidän itseäni ääriäni myöten analyyttisenä ihmisenä, joka tekee päätökset rationaalisten faktojen perusteella. Tiiminjäsenten valinnassa luottaisin itse faktoihin ja tiiminjäsenten kompetensseihin. Tällainen faktoihin pohjautuva tiiminjäsenten valinta saattaa tuottaa tuloksena tiimin, joka on kyvyiltään tähtitiimi, mutta jonka henkilökemiat eivät sopi yhteen keskeneen. Olen varma, että itselläni on paljon kehitettävää nimenomaan tunneälyn puolella.

Varis nosti luennollaan esille myös palkitsemisen merkityksen. Palkitsemisella Varis ei tarkoittanut niinkään rahallisen palkinnon antamista vaan pikemminkin tunnustuksen antamista. Hän kertoi kaksi esimerkkiä yksinkertaisesta palkitsemisesta. Ensimmäisessä esimerkissä Varis lupasi ostaa ohjelmistokehittäjälle palkinnoksi kossupullon, jos tämä tekisi valmiiksi halutun ominaisuuden. Toisessa esimerkissä hän kertoi tavasta, jossa tiimi valitsi joukostaan viikon työntekijän.

Mielestäni kumpikaan esimerkeistä ei kuitenkaan ole paras mahdollinen esimerkki onnistuneesta palkitsemisesta. Erityisesti ensimmäinen esimerkki on mielestäni jopa hieman huolestuttava. Miten pääsi syntymään tilanne, jossa kehittäjän piti käytännössä "kiristää" kossupullo esimieheltään, jotta tämä suostui tekemään ominaisuuden? Oliko Varis kentien erityisen huono antamaan alaisille tunnustusta ja tästä syystä alainen joutui hieman radikaalein keinoin muistuttamaan esimiestään tästä? Vai oliko kehittäjä kenties kyvytön ymmärtämään kokonaisuutta ja näkemään ominaisuuden tärkeyttä tuotteen kannalta? Tällainen etukäteen palkitseminen ei varmastikaan voi muodostua pitkäaikaiseksi tavaksi, mutta onneksi tämä oli ymmärtääkseni yksittäistapaus.

Toisessa esimerkissä positiivista oli se, että palkinnon saajan valitsi tiimi keskuudestaan. Toisaalta se, että palkinto oli viikottainen rutiini vähentää mielestäni palkinnon merkitystä. Olen sitä mieltä, että palkitseminen toimii parhaiten silloin kun siihen on aihetta, ei silloin kun on sen aika. Itse en tästä syystä käyttäisi viikottaista palkitsemista vaan palkitsisin kun kehittäjät yltävät erityisen upeisiin saavutuksiin.

Mielestäni eräs luennon ajatuksia herättävimmistä pointeista oli Variksen toteamus, jonka mukaan ihmisiä ei voi motivoida. Alaisia ei voi siis käskeä motivoitumaan. Jokaisen on itsensä motivoitava itsensä. Se mitä organisaatio ja esimies voivat asian hyväksi tehdä, on luoda ympäristö, jossa on helppo motivoitua.

% Feelings... how they effect

% Meeting a change, need of the change
% How it's affecting me? 20 mes?

% Tunneäly

% Palkitseminen, julkinen palkitseminen

% best worker of the week?

% Can not motivate people, create an environment

\section{Laura Snellman-Junna: Agile-valmennus ja johtajuus ketterässä kehityksessä}



% \bibliographystyle{plainnat}
% \bibliography{ref}

\end{document}