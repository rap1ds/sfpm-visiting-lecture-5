\documentclass[a4paper]{article}

% Encoodaus, joka sopii suomenkielellä (esim. ä ja ö)
\usepackage[utf8]{inputenc}
\usepackage[T1]{fontenc}

% Suomenkielinen tavutus
\usepackage[finnish]{babel}

% Viitteet
\usepackage{natbib}

% Otsikkojen päätteetön fontti
\usepackage{sectsty}
\allsectionsfont{\sffamily\large}

% Viitteiden merkit
\bibpunct{(}{)}{;}{a}{,}{,}

% page counting, header/footer
\usepackage{fancyhdr}
\usepackage{lastpage}
\usepackage{ifthen}
\pagestyle{fancy}
\lhead{\footnotesize \parbox{11cm}{} }
\rhead{\footnotesize \parbox{11cm}{} }
\lfoot{\footnotesize \parbox{11cm}{\ifthenelse{\thepage=\pageref{LastPage}}{Created with Dippa Editor \\ http://mkos.futupeople.com/dippa}{}}}
\renewcommand{\headrulewidth}{0.0pt}

\begin{document}

\title{\small T-76.5612 Software Project Management \\ Vierailuluento \\ \huge NSN \& Reaktor}
\date{27.2.2012}
\author{Mikko Koski \\ mikko.koski@aalto.fi \\ 66467F}
\maketitle

\normalsize

\section{Johdanto}

Luentovierailu suuntaitui tällä kertaa ohjelmistoyritys Reaktorin tiloihin. Luennoitsijoina toimivat Nokia Siemens Networkistä Maija-Riitta Varis sekä Reaktorilta Laura Snellman-Junna. Variksen luennon aiheena oli johtajuus, johtaminen sekä tiimihengen luonti ja Snellman-Junnann aiheena Agile-valmennus ja johtajuus ketterässä kehityksessä.

\section{Maija-Riitta Varis}

Maija-Riitta Varis kertoi luennolla kokemuksiaan toimimisesta ihmisten johtajana Nokia Siemens Networksillä. Luennon alkupuolella Varis puhui muutoksista, joita johtaja joutuu kohtaamaan työelämässä. Hän korosti, että muutokselle pitää aina olla järkevä syy, jotta muutoksen pystyy perustelemaan alaisilleen ja jotta alaiset voi motivoida muutoksen taakse.

Varis korosti luennollaan myös tunneälyn merkitystä. Hän suositteli aiheesta paria kirjaa sekä kertoi itse käyttävänsä tunneälyä johtamistehtävissä, esimerkiksi tiimijäsenten valinnassa. Tämä oli mielestäni mielenkiintoista kuultavaa, sillä minusta tuntuu, että Varis suhtautuu ihmisten johtamiseen hieman eri tavalla kuin mitä itse suhtautuisin. Pidän itseäni ääriäni myöten analyyttisenä ihmisenä, joka tekee päätökset rationaalisten faktojen perusteella. Tiiminjäsenten valinnassa luottaisin itse faktoihin ja tiiminjäsenten kompetensseihin. Tällainen faktoihin pohjautuva tiiminjäsenten valinta saattaa tuottaa tuloksena tiimin, joka on kyvyiltään tähtitiimi, mutta jonka henkilökemiat eivät sopi yhteen keskeneen. Olen varma, että itselläni on paljon kehitettävää nimenomaan tunneälyn puolella.



% Feelings... how they effect

% Meeting a change, need of the change
% How it's affecting me? 20 mes?

% Tunneäly

% Palkitseminen, julkinen palkitseminen

% best worker of the week?

% Can not motivate people, create an environment

\section{Laura Snellman-Junna: Agile-valmennus ja johtajuus ketterässä kehityksessä}



\bibliographystyle{plainnat}
\bibliography{ref}

\end{document}